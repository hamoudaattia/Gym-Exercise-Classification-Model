\documentclass[11pt]{article}

    \usepackage[breakable]{tcolorbox}
    \usepackage{parskip} % Stop auto-indenting (to mimic markdown behaviour)
    

    % Basic figure setup, for now with no caption control since it's done
    % automatically by Pandoc (which extracts ![](path) syntax from Markdown).
    \usepackage{graphicx}
    % Maintain compatibility with old templates. Remove in nbconvert 6.0
    \let\Oldincludegraphics\includegraphics
    % Ensure that by default, figures have no caption (until we provide a
    % proper Figure object with a Caption API and a way to capture that
    % in the conversion process - todo).
    \usepackage{caption}
    \DeclareCaptionFormat{nocaption}{}
    \captionsetup{format=nocaption,aboveskip=0pt,belowskip=0pt}

    \usepackage{float}
    \floatplacement{figure}{H} % forces figures to be placed at the correct location
    \usepackage{xcolor} % Allow colors to be defined
    \usepackage{enumerate} % Needed for markdown enumerations to work
    \usepackage{geometry} % Used to adjust the document margins
    \usepackage{amsmath} % Equations
    \usepackage{amssymb} % Equations
    \usepackage{textcomp} % defines textquotesingle
    % Hack from http://tex.stackexchange.com/a/47451/13684:
    \AtBeginDocument{%
        \def\PYZsq{\textquotesingle}% Upright quotes in Pygmentized code
    }
    \usepackage{upquote} % Upright quotes for verbatim code
    \usepackage{eurosym} % defines \euro

    \usepackage{iftex}
    \ifPDFTeX
        \usepackage[T1]{fontenc}
        \IfFileExists{alphabeta.sty}{
              \usepackage{alphabeta}
          }{
              \usepackage[mathletters]{ucs}
              \usepackage[utf8x]{inputenc}
          }
    \else
        \usepackage{fontspec}
        \usepackage{unicode-math}
    \fi

    \usepackage{fancyvrb} % verbatim replacement that allows latex
    \usepackage{grffile} % extends the file name processing of package graphics 
                         % to support a larger range
    \makeatletter % fix for old versions of grffile with XeLaTeX
    \@ifpackagelater{grffile}{2019/11/01}
    {
      % Do nothing on new versions
    }
    {
      \def\Gread@@xetex#1{%
        \IfFileExists{"\Gin@base".bb}%
        {\Gread@eps{\Gin@base.bb}}%
        {\Gread@@xetex@aux#1}%
      }
    }
    \makeatother
    \usepackage[Export]{adjustbox} % Used to constrain images to a maximum size
    \adjustboxset{max size={0.9\linewidth}{0.9\paperheight}}

    % The hyperref package gives us a pdf with properly built
    % internal navigation ('pdf bookmarks' for the table of contents,
    % internal cross-reference links, web links for URLs, etc.)
    \usepackage{hyperref}
    % The default LaTeX title has an obnoxious amount of whitespace. By default,
    % titling removes some of it. It also provides customization options.
    \usepackage{titling}
    \usepackage{longtable} % longtable support required by pandoc >1.10
    \usepackage{booktabs}  % table support for pandoc > 1.12.2
    \usepackage{array}     % table support for pandoc >= 2.11.3
    \usepackage{calc}      % table minipage width calculation for pandoc >= 2.11.1
    \usepackage[inline]{enumitem} % IRkernel/repr support (it uses the enumerate* environment)
    \usepackage[normalem]{ulem} % ulem is needed to support strikethroughs (\sout)
                                % normalem makes italics be italics, not underlines
    \usepackage{mathrsfs}
    

    
    % Colors for the hyperref package
    \definecolor{urlcolor}{rgb}{0,.145,.698}
    \definecolor{linkcolor}{rgb}{.71,0.21,0.01}
    \definecolor{citecolor}{rgb}{.12,.54,.11}

    % ANSI colors
    \definecolor{ansi-black}{HTML}{3E424D}
    \definecolor{ansi-black-intense}{HTML}{282C36}
    \definecolor{ansi-red}{HTML}{E75C58}
    \definecolor{ansi-red-intense}{HTML}{B22B31}
    \definecolor{ansi-green}{HTML}{00A250}
    \definecolor{ansi-green-intense}{HTML}{007427}
    \definecolor{ansi-yellow}{HTML}{DDB62B}
    \definecolor{ansi-yellow-intense}{HTML}{B27D12}
    \definecolor{ansi-blue}{HTML}{208FFB}
    \definecolor{ansi-blue-intense}{HTML}{0065CA}
    \definecolor{ansi-magenta}{HTML}{D160C4}
    \definecolor{ansi-magenta-intense}{HTML}{A03196}
    \definecolor{ansi-cyan}{HTML}{60C6C8}
    \definecolor{ansi-cyan-intense}{HTML}{258F8F}
    \definecolor{ansi-white}{HTML}{C5C1B4}
    \definecolor{ansi-white-intense}{HTML}{A1A6B2}
    \definecolor{ansi-default-inverse-fg}{HTML}{FFFFFF}
    \definecolor{ansi-default-inverse-bg}{HTML}{000000}

    % common color for the border for error outputs.
    \definecolor{outerrorbackground}{HTML}{FFDFDF}

    % commands and environments needed by pandoc snippets
    % extracted from the output of `pandoc -s`
    \providecommand{\tightlist}{%
      \setlength{\itemsep}{0pt}\setlength{\parskip}{0pt}}
    \DefineVerbatimEnvironment{Highlighting}{Verbatim}{commandchars=\\\{\}}
    % Add ',fontsize=\small' for more characters per line
    \newenvironment{Shaded}{}{}
    \newcommand{\KeywordTok}[1]{\textcolor[rgb]{0.00,0.44,0.13}{\textbf{{#1}}}}
    \newcommand{\DataTypeTok}[1]{\textcolor[rgb]{0.56,0.13,0.00}{{#1}}}
    \newcommand{\DecValTok}[1]{\textcolor[rgb]{0.25,0.63,0.44}{{#1}}}
    \newcommand{\BaseNTok}[1]{\textcolor[rgb]{0.25,0.63,0.44}{{#1}}}
    \newcommand{\FloatTok}[1]{\textcolor[rgb]{0.25,0.63,0.44}{{#1}}}
    \newcommand{\CharTok}[1]{\textcolor[rgb]{0.25,0.44,0.63}{{#1}}}
    \newcommand{\StringTok}[1]{\textcolor[rgb]{0.25,0.44,0.63}{{#1}}}
    \newcommand{\CommentTok}[1]{\textcolor[rgb]{0.38,0.63,0.69}{\textit{{#1}}}}
    \newcommand{\OtherTok}[1]{\textcolor[rgb]{0.00,0.44,0.13}{{#1}}}
    \newcommand{\AlertTok}[1]{\textcolor[rgb]{1.00,0.00,0.00}{\textbf{{#1}}}}
    \newcommand{\FunctionTok}[1]{\textcolor[rgb]{0.02,0.16,0.49}{{#1}}}
    \newcommand{\RegionMarkerTok}[1]{{#1}}
    \newcommand{\ErrorTok}[1]{\textcolor[rgb]{1.00,0.00,0.00}{\textbf{{#1}}}}
    \newcommand{\NormalTok}[1]{{#1}}
    
    % Additional commands for more recent versions of Pandoc
    \newcommand{\ConstantTok}[1]{\textcolor[rgb]{0.53,0.00,0.00}{{#1}}}
    \newcommand{\SpecialCharTok}[1]{\textcolor[rgb]{0.25,0.44,0.63}{{#1}}}
    \newcommand{\VerbatimStringTok}[1]{\textcolor[rgb]{0.25,0.44,0.63}{{#1}}}
    \newcommand{\SpecialStringTok}[1]{\textcolor[rgb]{0.73,0.40,0.53}{{#1}}}
    \newcommand{\ImportTok}[1]{{#1}}
    \newcommand{\DocumentationTok}[1]{\textcolor[rgb]{0.73,0.13,0.13}{\textit{{#1}}}}
    \newcommand{\AnnotationTok}[1]{\textcolor[rgb]{0.38,0.63,0.69}{\textbf{\textit{{#1}}}}}
    \newcommand{\CommentVarTok}[1]{\textcolor[rgb]{0.38,0.63,0.69}{\textbf{\textit{{#1}}}}}
    \newcommand{\VariableTok}[1]{\textcolor[rgb]{0.10,0.09,0.49}{{#1}}}
    \newcommand{\ControlFlowTok}[1]{\textcolor[rgb]{0.00,0.44,0.13}{\textbf{{#1}}}}
    \newcommand{\OperatorTok}[1]{\textcolor[rgb]{0.40,0.40,0.40}{{#1}}}
    \newcommand{\BuiltInTok}[1]{{#1}}
    \newcommand{\ExtensionTok}[1]{{#1}}
    \newcommand{\PreprocessorTok}[1]{\textcolor[rgb]{0.74,0.48,0.00}{{#1}}}
    \newcommand{\AttributeTok}[1]{\textcolor[rgb]{0.49,0.56,0.16}{{#1}}}
    \newcommand{\InformationTok}[1]{\textcolor[rgb]{0.38,0.63,0.69}{\textbf{\textit{{#1}}}}}
    \newcommand{\WarningTok}[1]{\textcolor[rgb]{0.38,0.63,0.69}{\textbf{\textit{{#1}}}}}
    
    
    % Define a nice break command that doesn't care if a line doesn't already
    % exist.
    \def\br{\hspace*{\fill} \\* }
    % Math Jax compatibility definitions
    \def\gt{>}
    \def\lt{<}
    \let\Oldtex\TeX
    \let\Oldlatex\LaTeX
    \renewcommand{\TeX}{\textrm{\Oldtex}}
    \renewcommand{\LaTeX}{\textrm{\Oldlatex}}
    % Document parameters
    % Document title
    \title{exercice classification ML}
    
    
    
    
    
% Pygments definitions
\makeatletter
\def\PY@reset{\let\PY@it=\relax \let\PY@bf=\relax%
    \let\PY@ul=\relax \let\PY@tc=\relax%
    \let\PY@bc=\relax \let\PY@ff=\relax}
\def\PY@tok#1{\csname PY@tok@#1\endcsname}
\def\PY@toks#1+{\ifx\relax#1\empty\else%
    \PY@tok{#1}\expandafter\PY@toks\fi}
\def\PY@do#1{\PY@bc{\PY@tc{\PY@ul{%
    \PY@it{\PY@bf{\PY@ff{#1}}}}}}}
\def\PY#1#2{\PY@reset\PY@toks#1+\relax+\PY@do{#2}}

\@namedef{PY@tok@w}{\def\PY@tc##1{\textcolor[rgb]{0.73,0.73,0.73}{##1}}}
\@namedef{PY@tok@c}{\let\PY@it=\textit\def\PY@tc##1{\textcolor[rgb]{0.24,0.48,0.48}{##1}}}
\@namedef{PY@tok@cp}{\def\PY@tc##1{\textcolor[rgb]{0.61,0.40,0.00}{##1}}}
\@namedef{PY@tok@k}{\let\PY@bf=\textbf\def\PY@tc##1{\textcolor[rgb]{0.00,0.50,0.00}{##1}}}
\@namedef{PY@tok@kp}{\def\PY@tc##1{\textcolor[rgb]{0.00,0.50,0.00}{##1}}}
\@namedef{PY@tok@kt}{\def\PY@tc##1{\textcolor[rgb]{0.69,0.00,0.25}{##1}}}
\@namedef{PY@tok@o}{\def\PY@tc##1{\textcolor[rgb]{0.40,0.40,0.40}{##1}}}
\@namedef{PY@tok@ow}{\let\PY@bf=\textbf\def\PY@tc##1{\textcolor[rgb]{0.67,0.13,1.00}{##1}}}
\@namedef{PY@tok@nb}{\def\PY@tc##1{\textcolor[rgb]{0.00,0.50,0.00}{##1}}}
\@namedef{PY@tok@nf}{\def\PY@tc##1{\textcolor[rgb]{0.00,0.00,1.00}{##1}}}
\@namedef{PY@tok@nc}{\let\PY@bf=\textbf\def\PY@tc##1{\textcolor[rgb]{0.00,0.00,1.00}{##1}}}
\@namedef{PY@tok@nn}{\let\PY@bf=\textbf\def\PY@tc##1{\textcolor[rgb]{0.00,0.00,1.00}{##1}}}
\@namedef{PY@tok@ne}{\let\PY@bf=\textbf\def\PY@tc##1{\textcolor[rgb]{0.80,0.25,0.22}{##1}}}
\@namedef{PY@tok@nv}{\def\PY@tc##1{\textcolor[rgb]{0.10,0.09,0.49}{##1}}}
\@namedef{PY@tok@no}{\def\PY@tc##1{\textcolor[rgb]{0.53,0.00,0.00}{##1}}}
\@namedef{PY@tok@nl}{\def\PY@tc##1{\textcolor[rgb]{0.46,0.46,0.00}{##1}}}
\@namedef{PY@tok@ni}{\let\PY@bf=\textbf\def\PY@tc##1{\textcolor[rgb]{0.44,0.44,0.44}{##1}}}
\@namedef{PY@tok@na}{\def\PY@tc##1{\textcolor[rgb]{0.41,0.47,0.13}{##1}}}
\@namedef{PY@tok@nt}{\let\PY@bf=\textbf\def\PY@tc##1{\textcolor[rgb]{0.00,0.50,0.00}{##1}}}
\@namedef{PY@tok@nd}{\def\PY@tc##1{\textcolor[rgb]{0.67,0.13,1.00}{##1}}}
\@namedef{PY@tok@s}{\def\PY@tc##1{\textcolor[rgb]{0.73,0.13,0.13}{##1}}}
\@namedef{PY@tok@sd}{\let\PY@it=\textit\def\PY@tc##1{\textcolor[rgb]{0.73,0.13,0.13}{##1}}}
\@namedef{PY@tok@si}{\let\PY@bf=\textbf\def\PY@tc##1{\textcolor[rgb]{0.64,0.35,0.47}{##1}}}
\@namedef{PY@tok@se}{\let\PY@bf=\textbf\def\PY@tc##1{\textcolor[rgb]{0.67,0.36,0.12}{##1}}}
\@namedef{PY@tok@sr}{\def\PY@tc##1{\textcolor[rgb]{0.64,0.35,0.47}{##1}}}
\@namedef{PY@tok@ss}{\def\PY@tc##1{\textcolor[rgb]{0.10,0.09,0.49}{##1}}}
\@namedef{PY@tok@sx}{\def\PY@tc##1{\textcolor[rgb]{0.00,0.50,0.00}{##1}}}
\@namedef{PY@tok@m}{\def\PY@tc##1{\textcolor[rgb]{0.40,0.40,0.40}{##1}}}
\@namedef{PY@tok@gh}{\let\PY@bf=\textbf\def\PY@tc##1{\textcolor[rgb]{0.00,0.00,0.50}{##1}}}
\@namedef{PY@tok@gu}{\let\PY@bf=\textbf\def\PY@tc##1{\textcolor[rgb]{0.50,0.00,0.50}{##1}}}
\@namedef{PY@tok@gd}{\def\PY@tc##1{\textcolor[rgb]{0.63,0.00,0.00}{##1}}}
\@namedef{PY@tok@gi}{\def\PY@tc##1{\textcolor[rgb]{0.00,0.52,0.00}{##1}}}
\@namedef{PY@tok@gr}{\def\PY@tc##1{\textcolor[rgb]{0.89,0.00,0.00}{##1}}}
\@namedef{PY@tok@ge}{\let\PY@it=\textit}
\@namedef{PY@tok@gs}{\let\PY@bf=\textbf}
\@namedef{PY@tok@gp}{\let\PY@bf=\textbf\def\PY@tc##1{\textcolor[rgb]{0.00,0.00,0.50}{##1}}}
\@namedef{PY@tok@go}{\def\PY@tc##1{\textcolor[rgb]{0.44,0.44,0.44}{##1}}}
\@namedef{PY@tok@gt}{\def\PY@tc##1{\textcolor[rgb]{0.00,0.27,0.87}{##1}}}
\@namedef{PY@tok@err}{\def\PY@bc##1{{\setlength{\fboxsep}{\string -\fboxrule}\fcolorbox[rgb]{1.00,0.00,0.00}{1,1,1}{\strut ##1}}}}
\@namedef{PY@tok@kc}{\let\PY@bf=\textbf\def\PY@tc##1{\textcolor[rgb]{0.00,0.50,0.00}{##1}}}
\@namedef{PY@tok@kd}{\let\PY@bf=\textbf\def\PY@tc##1{\textcolor[rgb]{0.00,0.50,0.00}{##1}}}
\@namedef{PY@tok@kn}{\let\PY@bf=\textbf\def\PY@tc##1{\textcolor[rgb]{0.00,0.50,0.00}{##1}}}
\@namedef{PY@tok@kr}{\let\PY@bf=\textbf\def\PY@tc##1{\textcolor[rgb]{0.00,0.50,0.00}{##1}}}
\@namedef{PY@tok@bp}{\def\PY@tc##1{\textcolor[rgb]{0.00,0.50,0.00}{##1}}}
\@namedef{PY@tok@fm}{\def\PY@tc##1{\textcolor[rgb]{0.00,0.00,1.00}{##1}}}
\@namedef{PY@tok@vc}{\def\PY@tc##1{\textcolor[rgb]{0.10,0.09,0.49}{##1}}}
\@namedef{PY@tok@vg}{\def\PY@tc##1{\textcolor[rgb]{0.10,0.09,0.49}{##1}}}
\@namedef{PY@tok@vi}{\def\PY@tc##1{\textcolor[rgb]{0.10,0.09,0.49}{##1}}}
\@namedef{PY@tok@vm}{\def\PY@tc##1{\textcolor[rgb]{0.10,0.09,0.49}{##1}}}
\@namedef{PY@tok@sa}{\def\PY@tc##1{\textcolor[rgb]{0.73,0.13,0.13}{##1}}}
\@namedef{PY@tok@sb}{\def\PY@tc##1{\textcolor[rgb]{0.73,0.13,0.13}{##1}}}
\@namedef{PY@tok@sc}{\def\PY@tc##1{\textcolor[rgb]{0.73,0.13,0.13}{##1}}}
\@namedef{PY@tok@dl}{\def\PY@tc##1{\textcolor[rgb]{0.73,0.13,0.13}{##1}}}
\@namedef{PY@tok@s2}{\def\PY@tc##1{\textcolor[rgb]{0.73,0.13,0.13}{##1}}}
\@namedef{PY@tok@sh}{\def\PY@tc##1{\textcolor[rgb]{0.73,0.13,0.13}{##1}}}
\@namedef{PY@tok@s1}{\def\PY@tc##1{\textcolor[rgb]{0.73,0.13,0.13}{##1}}}
\@namedef{PY@tok@mb}{\def\PY@tc##1{\textcolor[rgb]{0.40,0.40,0.40}{##1}}}
\@namedef{PY@tok@mf}{\def\PY@tc##1{\textcolor[rgb]{0.40,0.40,0.40}{##1}}}
\@namedef{PY@tok@mh}{\def\PY@tc##1{\textcolor[rgb]{0.40,0.40,0.40}{##1}}}
\@namedef{PY@tok@mi}{\def\PY@tc##1{\textcolor[rgb]{0.40,0.40,0.40}{##1}}}
\@namedef{PY@tok@il}{\def\PY@tc##1{\textcolor[rgb]{0.40,0.40,0.40}{##1}}}
\@namedef{PY@tok@mo}{\def\PY@tc##1{\textcolor[rgb]{0.40,0.40,0.40}{##1}}}
\@namedef{PY@tok@ch}{\let\PY@it=\textit\def\PY@tc##1{\textcolor[rgb]{0.24,0.48,0.48}{##1}}}
\@namedef{PY@tok@cm}{\let\PY@it=\textit\def\PY@tc##1{\textcolor[rgb]{0.24,0.48,0.48}{##1}}}
\@namedef{PY@tok@cpf}{\let\PY@it=\textit\def\PY@tc##1{\textcolor[rgb]{0.24,0.48,0.48}{##1}}}
\@namedef{PY@tok@c1}{\let\PY@it=\textit\def\PY@tc##1{\textcolor[rgb]{0.24,0.48,0.48}{##1}}}
\@namedef{PY@tok@cs}{\let\PY@it=\textit\def\PY@tc##1{\textcolor[rgb]{0.24,0.48,0.48}{##1}}}

\def\PYZbs{\char`\\}
\def\PYZus{\char`\_}
\def\PYZob{\char`\{}
\def\PYZcb{\char`\}}
\def\PYZca{\char`\^}
\def\PYZam{\char`\&}
\def\PYZlt{\char`\<}
\def\PYZgt{\char`\>}
\def\PYZsh{\char`\#}
\def\PYZpc{\char`\%}
\def\PYZdl{\char`\$}
\def\PYZhy{\char`\-}
\def\PYZsq{\char`\'}
\def\PYZdq{\char`\"}
\def\PYZti{\char`\~}
% for compatibility with earlier versions
\def\PYZat{@}
\def\PYZlb{[}
\def\PYZrb{]}
\makeatother


    % For linebreaks inside Verbatim environment from package fancyvrb. 
    \makeatletter
        \newbox\Wrappedcontinuationbox 
        \newbox\Wrappedvisiblespacebox 
        \newcommand*\Wrappedvisiblespace {\textcolor{red}{\textvisiblespace}} 
        \newcommand*\Wrappedcontinuationsymbol {\textcolor{red}{\llap{\tiny$\m@th\hookrightarrow$}}} 
        \newcommand*\Wrappedcontinuationindent {3ex } 
        \newcommand*\Wrappedafterbreak {\kern\Wrappedcontinuationindent\copy\Wrappedcontinuationbox} 
        % Take advantage of the already applied Pygments mark-up to insert 
        % potential linebreaks for TeX processing. 
        %        {, <, #, %, $, ' and ": go to next line. 
        %        _, }, ^, &, >, - and ~: stay at end of broken line. 
        % Use of \textquotesingle for straight quote. 
        \newcommand*\Wrappedbreaksatspecials {% 
            \def\PYGZus{\discretionary{\char`\_}{\Wrappedafterbreak}{\char`\_}}% 
            \def\PYGZob{\discretionary{}{\Wrappedafterbreak\char`\{}{\char`\{}}% 
            \def\PYGZcb{\discretionary{\char`\}}{\Wrappedafterbreak}{\char`\}}}% 
            \def\PYGZca{\discretionary{\char`\^}{\Wrappedafterbreak}{\char`\^}}% 
            \def\PYGZam{\discretionary{\char`\&}{\Wrappedafterbreak}{\char`\&}}% 
            \def\PYGZlt{\discretionary{}{\Wrappedafterbreak\char`\<}{\char`\<}}% 
            \def\PYGZgt{\discretionary{\char`\>}{\Wrappedafterbreak}{\char`\>}}% 
            \def\PYGZsh{\discretionary{}{\Wrappedafterbreak\char`\#}{\char`\#}}% 
            \def\PYGZpc{\discretionary{}{\Wrappedafterbreak\char`\%}{\char`\%}}% 
            \def\PYGZdl{\discretionary{}{\Wrappedafterbreak\char`\$}{\char`\$}}% 
            \def\PYGZhy{\discretionary{\char`\-}{\Wrappedafterbreak}{\char`\-}}% 
            \def\PYGZsq{\discretionary{}{\Wrappedafterbreak\textquotesingle}{\textquotesingle}}% 
            \def\PYGZdq{\discretionary{}{\Wrappedafterbreak\char`\"}{\char`\"}}% 
            \def\PYGZti{\discretionary{\char`\~}{\Wrappedafterbreak}{\char`\~}}% 
        } 
        % Some characters . , ; ? ! / are not pygmentized. 
        % This macro makes them "active" and they will insert potential linebreaks 
        \newcommand*\Wrappedbreaksatpunct {% 
            \lccode`\~`\.\lowercase{\def~}{\discretionary{\hbox{\char`\.}}{\Wrappedafterbreak}{\hbox{\char`\.}}}% 
            \lccode`\~`\,\lowercase{\def~}{\discretionary{\hbox{\char`\,}}{\Wrappedafterbreak}{\hbox{\char`\,}}}% 
            \lccode`\~`\;\lowercase{\def~}{\discretionary{\hbox{\char`\;}}{\Wrappedafterbreak}{\hbox{\char`\;}}}% 
            \lccode`\~`\:\lowercase{\def~}{\discretionary{\hbox{\char`\:}}{\Wrappedafterbreak}{\hbox{\char`\:}}}% 
            \lccode`\~`\?\lowercase{\def~}{\discretionary{\hbox{\char`\?}}{\Wrappedafterbreak}{\hbox{\char`\?}}}% 
            \lccode`\~`\!\lowercase{\def~}{\discretionary{\hbox{\char`\!}}{\Wrappedafterbreak}{\hbox{\char`\!}}}% 
            \lccode`\~`\/\lowercase{\def~}{\discretionary{\hbox{\char`\/}}{\Wrappedafterbreak}{\hbox{\char`\/}}}% 
            \catcode`\.\active
            \catcode`\,\active 
            \catcode`\;\active
            \catcode`\:\active
            \catcode`\?\active
            \catcode`\!\active
            \catcode`\/\active 
            \lccode`\~`\~ 	
        }
    \makeatother

    \let\OriginalVerbatim=\Verbatim
    \makeatletter
    \renewcommand{\Verbatim}[1][1]{%
        %\parskip\z@skip
        \sbox\Wrappedcontinuationbox {\Wrappedcontinuationsymbol}%
        \sbox\Wrappedvisiblespacebox {\FV@SetupFont\Wrappedvisiblespace}%
        \def\FancyVerbFormatLine ##1{\hsize\linewidth
            \vtop{\raggedright\hyphenpenalty\z@\exhyphenpenalty\z@
                \doublehyphendemerits\z@\finalhyphendemerits\z@
                \strut ##1\strut}%
        }%
        % If the linebreak is at a space, the latter will be displayed as visible
        % space at end of first line, and a continuation symbol starts next line.
        % Stretch/shrink are however usually zero for typewriter font.
        \def\FV@Space {%
            \nobreak\hskip\z@ plus\fontdimen3\font minus\fontdimen4\font
            \discretionary{\copy\Wrappedvisiblespacebox}{\Wrappedafterbreak}
            {\kern\fontdimen2\font}%
        }%
        
        % Allow breaks at special characters using \PYG... macros.
        \Wrappedbreaksatspecials
        % Breaks at punctuation characters . , ; ? ! and / need catcode=\active 	
        \OriginalVerbatim[#1,codes*=\Wrappedbreaksatpunct]%
    }
    \makeatother

    % Exact colors from NB
    \definecolor{incolor}{HTML}{303F9F}
    \definecolor{outcolor}{HTML}{D84315}
    \definecolor{cellborder}{HTML}{CFCFCF}
    \definecolor{cellbackground}{HTML}{F7F7F7}
    
    % prompt
    \makeatletter
    \newcommand{\boxspacing}{\kern\kvtcb@left@rule\kern\kvtcb@boxsep}
    \makeatother
    \newcommand{\prompt}[4]{
        {\ttfamily\llap{{\color{#2}[#3]:\hspace{3pt}#4}}\vspace{-\baselineskip}}
    }
    

    
    % Prevent overflowing lines due to hard-to-break entities
    \sloppy 
    % Setup hyperref package
    \hypersetup{
      breaklinks=true,  % so long urls are correctly broken across lines
      colorlinks=true,
      urlcolor=urlcolor,
      linkcolor=linkcolor,
      citecolor=citecolor,
      }
    % Slightly bigger margins than the latex defaults
    
    \geometry{verbose,tmargin=1in,bmargin=1in,lmargin=1in,rmargin=1in}
    
    

\begin{document}
    
    \maketitle
    
    

    
    \begin{tcolorbox}[breakable, size=fbox, boxrule=1pt, pad at break*=1mm,colback=cellbackground, colframe=cellborder]
\prompt{In}{incolor}{152}{\boxspacing}
\begin{Verbatim}[commandchars=\\\{\}]
\PY{k}{using} \PY{n}{CSV}
\PY{k}{using} \PY{n}{DataFrames}
\PY{k}{using} \PY{n}{MLJ}
\end{Verbatim}
\end{tcolorbox}

    \begin{tcolorbox}[breakable, size=fbox, boxrule=1pt, pad at break*=1mm,colback=cellbackground, colframe=cellborder]
\prompt{In}{incolor}{153}{\boxspacing}
\begin{Verbatim}[commandchars=\\\{\}]
\PY{c}{\PYZsh{} Read the CSV file into a DataFrame}
\PY{n}{df} \PY{o}{=} \PY{n}{CSV}\PY{o}{.}\PY{n}{read}\PY{p}{(}\PY{l+s}{\PYZdq{}}\PY{l+s}{megaGymDataset.csv}\PY{l+s}{\PYZdq{}}\PY{p}{,} \PY{n}{DataFrame}\PY{p}{)}

\PY{c}{\PYZsh{} Display summary statistics of the DataFrame}
\PY{n}{describe}\PY{p}{(}\PY{n}{df}\PY{p}{)}

\PY{c}{\PYZsh{} Get the number of rows and columns in the DataFrame}
\PY{n}{nrow}\PY{p}{(}\PY{n}{df}\PY{p}{)}\PY{p}{,} \PY{n}{ncol}\PY{p}{(}\PY{n}{df}\PY{p}{)}
\end{Verbatim}
\end{tcolorbox}

            \begin{tcolorbox}[breakable, size=fbox, boxrule=.5pt, pad at break*=1mm, opacityfill=0]
\prompt{Out}{outcolor}{153}{\boxspacing}
\begin{Verbatim}[commandchars=\\\{\}]
(2918, 9)
\end{Verbatim}
\end{tcolorbox}
        
    \begin{tcolorbox}[breakable, size=fbox, boxrule=1pt, pad at break*=1mm,colback=cellbackground, colframe=cellborder]
\prompt{In}{incolor}{154}{\boxspacing}
\begin{Verbatim}[commandchars=\\\{\}]
\PY{c}{\PYZsh{} Select the two columns from the original DataFrame}
\PY{n}{selected\PYZus{}columns} \PY{o}{=} \PY{n}{df}\PY{p}{[}\PY{o}{:}\PY{p}{,} \PY{p}{[}\PY{l+s+ss}{:Title}\PY{p}{,}  \PY{l+s+ss}{:BodyPart}\PY{p}{]}\PY{p}{]}
\PY{c}{\PYZsh{} Create a new DataFrame with the selected columns}
\PY{n}{new\PYZus{}df} \PY{o}{=} \PY{n}{DataFrame}\PY{p}{(}\PY{n}{selected\PYZus{}columns}\PY{p}{)}
\PY{n}{nrow}\PY{p}{(}\PY{n}{new\PYZus{}df}\PY{p}{)}\PY{p}{,} \PY{n}{ncol}\PY{p}{(}\PY{n}{new\PYZus{}df}\PY{p}{)}
\end{Verbatim}
\end{tcolorbox}

            \begin{tcolorbox}[breakable, size=fbox, boxrule=.5pt, pad at break*=1mm, opacityfill=0]
\prompt{Out}{outcolor}{154}{\boxspacing}
\begin{Verbatim}[commandchars=\\\{\}]
(2918, 2)
\end{Verbatim}
\end{tcolorbox}
        
    \begin{tcolorbox}[breakable, size=fbox, boxrule=1pt, pad at break*=1mm,colback=cellbackground, colframe=cellborder]
\prompt{In}{incolor}{155}{\boxspacing}
\begin{Verbatim}[commandchars=\\\{\}]
\PY{n}{schema}\PY{p}{(}\PY{n}{new\PYZus{}df}\PY{p}{)}
\end{Verbatim}
\end{tcolorbox}

            \begin{tcolorbox}[breakable, size=fbox, boxrule=.5pt, pad at break*=1mm, opacityfill=0]
\prompt{Out}{outcolor}{155}{\boxspacing}
\begin{Verbatim}[commandchars=\\\{\}]
┌──────────┬──────────┬──────────┐
│ names    │ scitypes │ types    │
├──────────┼──────────┼──────────┤
│ Title    │ Textual  │ String   │
│ BodyPart │ Textual  │ String15 │
└──────────┴──────────┴──────────┘

\end{Verbatim}
\end{tcolorbox}
        
    \begin{tcolorbox}[breakable, size=fbox, boxrule=1pt, pad at break*=1mm,colback=cellbackground, colframe=cellborder]
\prompt{In}{incolor}{156}{\boxspacing}
\begin{Verbatim}[commandchars=\\\{\}]
\PY{k}{using} \PY{n}{Flux}

\PY{n}{X} \PY{o}{=} \PY{n}{new\PYZus{}df}\PY{o}{.}\PY{n}{Title}  \PY{c}{\PYZsh{} Exercise names}
\PY{n}{y} \PY{o}{=} \PY{n}{new\PYZus{}df}\PY{o}{.}\PY{n}{BodyPart}   \PY{c}{\PYZsh{} Target body parts}
\end{Verbatim}
\end{tcolorbox}

            \begin{tcolorbox}[breakable, size=fbox, boxrule=.5pt, pad at break*=1mm, opacityfill=0]
\prompt{Out}{outcolor}{156}{\boxspacing}
\begin{Verbatim}[commandchars=\\\{\}]
2918-element PooledArrays.PooledVector\{String15, UInt32, Vector\{UInt32\}\}:
 "Abdominals"
 "Abdominals"
 "Abdominals"
 "Abdominals"
 "Abdominals"
 "Abdominals"
 "Abdominals"
 "Abdominals"
 "Abdominals"
 "Abdominals"
 "Abdominals"
 "Abdominals"
 "Abdominals"
 ⋮
 "Triceps"
 "Triceps"
 "Triceps"
 "Triceps"
 "Triceps"
 "Triceps"
 "Triceps"
 "Triceps"
 "Triceps"
 "Triceps"
 "Triceps"
 "Triceps"
\end{Verbatim}
\end{tcolorbox}
        
    \begin{tcolorbox}[breakable, size=fbox, boxrule=1pt, pad at break*=1mm,colback=cellbackground, colframe=cellborder]
\prompt{In}{incolor}{157}{\boxspacing}
\begin{Verbatim}[commandchars=\\\{\}]
\PY{c}{\PYZsh{} Step 2: Feature Extraction}
\PY{c}{\PYZsh{} In this simple example, we\PYZsq{}ll use one\PYZhy{}hot encoding for the exercise names}
\PY{n}{X\PYZus{}encoded} \PY{o}{=} \PY{n}{Flux}\PY{o}{.}\PY{n}{onehotbatch}\PY{p}{(}\PY{n}{X}\PY{p}{,} \PY{n}{sort}\PY{p}{(}\PY{n}{unique}\PY{p}{(}\PY{n}{X}\PY{p}{)}\PY{p}{)}\PY{p}{)} \PY{o}{|\PYZgt{}} \PY{k+kt}{Matrix}\PY{p}{\PYZob{}}\PY{k+kt}{Float32}\PY{p}{\PYZcb{}}
\PY{n}{X\PYZus{}unique} \PY{o}{=} \PY{n}{sort}\PY{p}{(}\PY{n}{unique}\PY{p}{(}\PY{n}{X}\PY{p}{)}\PY{p}{)}
\end{Verbatim}
\end{tcolorbox}

            \begin{tcolorbox}[breakable, size=fbox, boxrule=.5pt, pad at break*=1mm, opacityfill=0]
\prompt{Out}{outcolor}{157}{\boxspacing}
\begin{Verbatim}[commandchars=\\\{\}]
2909-element Vector\{String\}:
 "1.5-rep push-up"
 "3/4 sit-up"
 "30 Arms BFR Close-Grip Push-Up"
 "30 Arms BFR Dumbbell Kick-Back"
 "30 Arms BFR High Cable Curl"
 "30 Arms BFR Machine Preacher Curl"
 "30 Arms Barbell Skullcrusher"
 "30 Arms Cable Concentration Curl"
 "30 Arms Cable Rope Hammer Curl"
 "30 Arms Cable Rope Overhead Triceps Extension"
 "30 Arms Cable Rope Push-Down"
 "30 Arms Cable Straight-Bar Curl"
 "30 Arms Cable Straight-Bar Push-Down"
 ⋮
 "Wide-grip bench press"
 "Wide-grip hands-elevated push-up"
 "Windmills"
 "World's greatest stretch"
 "Wrist Circles"
 "Wrist Roller"
 "X-body V-up"
 "Yates Row"
 "Yates Row Reverse Grip"
 "Yoga plex"
 "Yoke Walk"
 "Zercher squat"
\end{Verbatim}
\end{tcolorbox}
        
    \begin{tcolorbox}[breakable, size=fbox, boxrule=1pt, pad at break*=1mm,colback=cellbackground, colframe=cellborder]
\prompt{In}{incolor}{158}{\boxspacing}
\begin{Verbatim}[commandchars=\\\{\}]
\PY{k}{using} \PY{n}{Statistics}
\PY{c}{\PYZsh{} Standardize the input features using z\PYZhy{}score normalization}
\PY{n}{X\PYZus{}standardized} \PY{o}{=} \PY{p}{(}\PY{n}{X\PYZus{}encoded} \PY{o}{.\PYZhy{}} \PY{n}{mean}\PY{p}{(}\PY{n}{X\PYZus{}encoded}\PY{p}{,} \PY{n}{dims}\PY{o}{=}\PY{l+m+mi}{1}\PY{p}{)}\PY{p}{)} \PY{o}{./} \PY{n}{std}\PY{p}{(}\PY{n}{X\PYZus{}encoded}\PY{p}{,} \PY{n}{dims}\PY{o}{=}\PY{l+m+mi}{1}\PY{p}{)}
\end{Verbatim}
\end{tcolorbox}

            \begin{tcolorbox}[breakable, size=fbox, boxrule=.5pt, pad at break*=1mm, opacityfill=0]
\prompt{Out}{outcolor}{158}{\boxspacing}
\begin{Verbatim}[commandchars=\\\{\}]
2909×2918 Matrix\{Float32\}:
 -0.0185408  -0.0185408  -0.0185408  …  -0.0185408  -0.0185408  -0.0185408
 -0.0185408  -0.0185408  -0.0185408     -0.0185408  -0.0185408  -0.0185408
 -0.0185408  -0.0185408  -0.0185408     -0.0185408  -0.0185408  -0.0185408
 -0.0185408  -0.0185408  -0.0185408     -0.0185408  -0.0185408  -0.0185408
 -0.0185408  -0.0185408  -0.0185408     -0.0185408  -0.0185408  -0.0185408
 -0.0185408  -0.0185408  -0.0185408  …  -0.0185408  -0.0185408  -0.0185408
 -0.0185408  -0.0185408  -0.0185408     -0.0185408  -0.0185408  -0.0185408
 -0.0185408  -0.0185408  -0.0185408     -0.0185408  -0.0185408  -0.0185408
 -0.0185408  -0.0185408  -0.0185408     -0.0185408  -0.0185408  -0.0185408
 -0.0185408  -0.0185408  -0.0185408     -0.0185408  -0.0185408  -0.0185408
 -0.0185408  -0.0185408  -0.0185408  …  -0.0185408  -0.0185408  -0.0185408
 -0.0185408  -0.0185408  -0.0185408     -0.0185408  -0.0185408  -0.0185408
 -0.0185408  -0.0185408  -0.0185408     -0.0185408  -0.0185408  -0.0185408
  ⋮                                  ⋱   ⋮
 -0.0185408  -0.0185408  -0.0185408     -0.0185408  -0.0185408  -0.0185408
 -0.0185408  -0.0185408  -0.0185408     -0.0185408  -0.0185408  -0.0185408
 -0.0185408  -0.0185408  -0.0185408     -0.0185408  -0.0185408  -0.0185408
 -0.0185408  -0.0185408  -0.0185408  …  -0.0185408  -0.0185408  -0.0185408
 -0.0185408  -0.0185408  -0.0185408     -0.0185408  -0.0185408  -0.0185408
 -0.0185408  -0.0185408  -0.0185408     -0.0185408  -0.0185408  -0.0185408
 -0.0185408  -0.0185408  -0.0185408     -0.0185408  -0.0185408  -0.0185408
 -0.0185408  -0.0185408  -0.0185408     -0.0185408  -0.0185408  -0.0185408
 -0.0185408  -0.0185408  -0.0185408  …  -0.0185408  -0.0185408  -0.0185408
 -0.0185408  -0.0185408  -0.0185408     -0.0185408  -0.0185408  -0.0185408
 -0.0185408  -0.0185408  -0.0185408     -0.0185408  -0.0185408  -0.0185408
 -0.0185408  -0.0185408  -0.0185408     -0.0185408  -0.0185408  -0.0185408
\end{Verbatim}
\end{tcolorbox}
        
    \begin{tcolorbox}[breakable, size=fbox, boxrule=1pt, pad at break*=1mm,colback=cellbackground, colframe=cellborder]
\prompt{In}{incolor}{159}{\boxspacing}
\begin{Verbatim}[commandchars=\\\{\}]
\PY{c}{\PYZsh{} Encode target labels as integers}
\PY{n}{label\PYZus{}mapping} \PY{o}{=} \PY{k+kt}{Dict}\PY{p}{(}\PY{n}{unique}\PY{p}{(}\PY{n}{y}\PY{p}{)} \PY{o}{.=\PYZgt{}} \PY{l+m+mi}{1}\PY{o}{:}\PY{n}{length}\PY{p}{(}\PY{n}{unique}\PY{p}{(}\PY{n}{y}\PY{p}{)}\PY{p}{)}\PY{p}{)}
\PY{n}{y\PYZus{}encoded} \PY{o}{=} \PY{p}{[}\PY{n}{label\PYZus{}mapping}\PY{p}{[}\PY{n}{label}\PY{p}{]} \PY{k}{for} \PY{n}{label} \PY{k}{in} \PY{n}{y}\PY{p}{]}
\end{Verbatim}
\end{tcolorbox}

            \begin{tcolorbox}[breakable, size=fbox, boxrule=.5pt, pad at break*=1mm, opacityfill=0]
\prompt{Out}{outcolor}{159}{\boxspacing}
\begin{Verbatim}[commandchars=\\\{\}]
2918-element Vector\{Int64\}:
  1
  1
  1
  1
  1
  1
  1
  1
  1
  1
  1
  1
  1
  ⋮
 17
 17
 17
 17
 17
 17
 17
 17
 17
 17
 17
 17
\end{Verbatim}
\end{tcolorbox}
        
    \begin{tcolorbox}[breakable, size=fbox, boxrule=1pt, pad at break*=1mm,colback=cellbackground, colframe=cellborder]
\prompt{In}{incolor}{160}{\boxspacing}
\begin{Verbatim}[commandchars=\\\{\}]
\PY{c}{\PYZsh{} Step 3: Model Training}
\PY{c}{\PYZsh{} Split the data into training and testing sets}
\PY{n}{data} \PY{o}{=} \PY{p}{[}\PY{p}{(}\PY{n}{x}\PY{p}{,} \PY{n}{y}\PY{p}{)} \PY{k}{for} \PY{p}{(}\PY{n}{x}\PY{p}{,} \PY{n}{y}\PY{p}{)} \PY{k}{in} \PY{n}{zip}\PY{p}{(}\PY{n}{eachrow}\PY{p}{(}\PY{n}{X\PYZus{}encoded}\PY{p}{)}\PY{p}{,} \PY{n}{y\PYZus{}encoded}\PY{p}{)}\PY{p}{]}
\PY{n}{train\PYZus{}size} \PY{o}{=} \PY{k+kt}{Int}\PY{p}{(}\PY{n}{round}\PY{p}{(}\PY{n}{length}\PY{p}{(}\PY{n}{data}\PY{p}{)} \PY{o}{*} \PY{l+m+mf}{0.8}\PY{p}{)}\PY{p}{)}
\PY{n}{train\PYZus{}data} \PY{o}{=} \PY{n}{data}\PY{p}{[}\PY{l+m+mi}{1}\PY{o}{:}\PY{n}{train\PYZus{}size}\PY{p}{]}
\PY{n}{test\PYZus{}data} \PY{o}{=} \PY{n}{data}\PY{p}{[}\PY{n}{train\PYZus{}size}\PY{o}{+}\PY{l+m+mi}{1}\PY{o}{:}\PY{k}{end}\PY{p}{]}
\end{Verbatim}
\end{tcolorbox}

            \begin{tcolorbox}[breakable, size=fbox, boxrule=.5pt, pad at break*=1mm, opacityfill=0]
\prompt{Out}{outcolor}{160}{\boxspacing}
\begin{Verbatim}[commandchars=\\\{\}]
582-element Vector\{Tuple\{SubArray\{Float32, 1, Matrix\{Float32\}, Tuple\{Int64,
Base.Slice\{Base.OneTo\{Int64\}\}\}, true\}, Int64\}\}:
 ([0.0, 0.0, 0.0, 0.0, 0.0, 0.0, 0.0, 0.0, 0.0, 0.0  …  0.0, 0.0, 0.0, 0.0, 0.0,
0.0, 0.0, 0.0, 0.0, 0.0], 15)
 ([0.0, 0.0, 0.0, 0.0, 0.0, 0.0, 0.0, 0.0, 0.0, 0.0  …  0.0, 0.0, 0.0, 0.0, 0.0,
0.0, 0.0, 0.0, 0.0, 0.0], 15)
 ([0.0, 0.0, 0.0, 0.0, 0.0, 0.0, 0.0, 0.0, 0.0, 0.0  …  0.0, 0.0, 0.0, 0.0, 0.0,
0.0, 0.0, 0.0, 0.0, 0.0], 15)
 ([0.0, 0.0, 0.0, 0.0, 0.0, 0.0, 0.0, 0.0, 0.0, 0.0  …  0.0, 0.0, 0.0, 0.0, 0.0,
0.0, 0.0, 0.0, 0.0, 0.0], 15)
 ([0.0, 0.0, 0.0, 0.0, 0.0, 0.0, 0.0, 0.0, 0.0, 0.0  …  0.0, 0.0, 0.0, 0.0, 0.0,
0.0, 0.0, 0.0, 0.0, 0.0], 15)
 ([0.0, 0.0, 0.0, 0.0, 0.0, 0.0, 0.0, 0.0, 0.0, 0.0  …  0.0, 0.0, 0.0, 0.0, 0.0,
0.0, 0.0, 0.0, 0.0, 0.0], 15)
 ([0.0, 0.0, 0.0, 0.0, 0.0, 0.0, 0.0, 0.0, 0.0, 0.0  …  0.0, 0.0, 0.0, 0.0, 0.0,
0.0, 0.0, 0.0, 0.0, 0.0], 15)
 ([0.0, 0.0, 0.0, 0.0, 0.0, 0.0, 0.0, 0.0, 0.0, 0.0  …  0.0, 0.0, 0.0, 0.0, 0.0,
0.0, 0.0, 0.0, 0.0, 0.0], 15)
 ([0.0, 0.0, 0.0, 0.0, 0.0, 0.0, 0.0, 0.0, 0.0, 0.0  …  0.0, 0.0, 0.0, 0.0, 0.0,
0.0, 0.0, 0.0, 0.0, 0.0], 15)
 ([0.0, 0.0, 0.0, 0.0, 0.0, 0.0, 0.0, 0.0, 0.0, 0.0  …  0.0, 0.0, 0.0, 0.0, 0.0,
0.0, 0.0, 0.0, 0.0, 0.0], 15)
 ([0.0, 0.0, 0.0, 0.0, 0.0, 0.0, 0.0, 0.0, 0.0, 0.0  …  0.0, 0.0, 0.0, 0.0, 0.0,
0.0, 0.0, 0.0, 0.0, 0.0], 15)
 ([0.0, 0.0, 0.0, 0.0, 0.0, 0.0, 0.0, 0.0, 0.0, 0.0  …  0.0, 0.0, 0.0, 0.0, 0.0,
0.0, 0.0, 0.0, 0.0, 0.0], 15)
 ([0.0, 0.0, 0.0, 0.0, 0.0, 0.0, 0.0, 0.0, 0.0, 0.0  …  0.0, 0.0, 0.0, 0.0, 0.0,
0.0, 0.0, 0.0, 0.0, 0.0], 15)
 ⋮
 ([0.0, 0.0, 0.0, 0.0, 0.0, 0.0, 0.0, 0.0, 0.0, 0.0  …  0.0, 0.0, 0.0, 0.0, 0.0,
0.0, 0.0, 0.0, 0.0, 0.0], 17)
 ([0.0, 0.0, 0.0, 0.0, 0.0, 0.0, 0.0, 0.0, 0.0, 0.0  …  0.0, 0.0, 0.0, 0.0, 0.0,
0.0, 0.0, 0.0, 0.0, 0.0], 17)
 ([0.0, 0.0, 0.0, 0.0, 0.0, 0.0, 0.0, 0.0, 0.0, 0.0  …  0.0, 0.0, 0.0, 0.0, 0.0,
0.0, 0.0, 0.0, 0.0, 0.0], 17)
 ([0.0, 0.0, 0.0, 0.0, 0.0, 0.0, 0.0, 0.0, 0.0, 0.0  …  0.0, 0.0, 0.0, 0.0, 0.0,
0.0, 0.0, 0.0, 0.0, 0.0], 17)
 ([0.0, 0.0, 0.0, 0.0, 0.0, 0.0, 0.0, 0.0, 0.0, 0.0  …  0.0, 0.0, 0.0, 0.0, 0.0,
0.0, 0.0, 0.0, 0.0, 0.0], 17)
 ([0.0, 0.0, 0.0, 0.0, 0.0, 0.0, 0.0, 0.0, 0.0, 0.0  …  0.0, 0.0, 0.0, 0.0, 0.0,
0.0, 0.0, 0.0, 0.0, 0.0], 17)
 ([0.0, 0.0, 0.0, 0.0, 0.0, 0.0, 0.0, 0.0, 0.0, 0.0  …  0.0, 0.0, 0.0, 0.0, 0.0,
0.0, 0.0, 0.0, 0.0, 0.0], 17)
 ([0.0, 0.0, 0.0, 0.0, 0.0, 0.0, 0.0, 0.0, 0.0, 0.0  …  0.0, 0.0, 0.0, 0.0, 0.0,
0.0, 0.0, 0.0, 0.0, 0.0], 17)
 ([0.0, 0.0, 0.0, 0.0, 0.0, 0.0, 0.0, 0.0, 0.0, 0.0  …  0.0, 0.0, 0.0, 0.0, 0.0,
0.0, 0.0, 0.0, 0.0, 0.0], 17)
 ([0.0, 0.0, 0.0, 0.0, 0.0, 0.0, 0.0, 0.0, 0.0, 0.0  …  0.0, 0.0, 0.0, 0.0, 0.0,
0.0, 0.0, 0.0, 0.0, 0.0], 17)
 ([0.0, 0.0, 0.0, 0.0, 0.0, 0.0, 0.0, 0.0, 0.0, 0.0  …  0.0, 0.0, 0.0, 0.0, 0.0,
0.0, 0.0, 0.0, 0.0, 0.0], 17)
 ([0.0, 0.0, 0.0, 0.0, 0.0, 0.0, 0.0, 0.0, 0.0, 0.0  …  0.0, 0.0, 0.0, 0.0, 0.0,
0.0, 0.0, 0.0, 0.0, 0.0], 17)
\end{Verbatim}
\end{tcolorbox}
        
    \begin{tcolorbox}[breakable, size=fbox, boxrule=1pt, pad at break*=1mm,colback=cellbackground, colframe=cellborder]
\prompt{In}{incolor}{176}{\boxspacing}
\begin{Verbatim}[commandchars=\\\{\}]
\PY{c}{\PYZsh{} Define the model architecture}
\PY{n}{input\PYZus{}size} \PY{o}{=} \PY{n}{size}\PY{p}{(}\PY{n}{X\PYZus{}encoded}\PY{p}{,} \PY{l+m+mi}{2}\PY{p}{)}
\PY{n}{model} \PY{o}{=} \PY{n}{Chain}\PY{p}{(}
    \PY{n}{Dense}\PY{p}{(}\PY{n}{input\PYZus{}size}\PY{p}{,} \PY{l+m+mi}{64}\PY{p}{,} \PY{n}{relu}\PY{p}{)}\PY{p}{,}
    \PY{n}{Dense}\PY{p}{(}\PY{l+m+mi}{64}\PY{p}{,} \PY{l+m+mi}{32}\PY{p}{,} \PY{n}{relu}\PY{p}{)}\PY{p}{,}
    \PY{n}{Dense}\PY{p}{(}\PY{l+m+mi}{32}\PY{p}{,} \PY{n}{length}\PY{p}{(}\PY{n}{label\PYZus{}mapping}\PY{p}{)}\PY{p}{)}
\PY{p}{)}
\end{Verbatim}
\end{tcolorbox}

            \begin{tcolorbox}[breakable, size=fbox, boxrule=.5pt, pad at break*=1mm, opacityfill=0]
\prompt{Out}{outcolor}{176}{\boxspacing}
\begin{Verbatim}[commandchars=\\\{\}]
Chain(
  Dense(2918 => 64, relu),              \textcolor{ansi-black-intense}{\# 186\_816 parameters}
  Dense(64 => 32, relu),                \textcolor{ansi-black-intense}{\# 2\_080 parameters}
  Dense(32 => 17),                      \textcolor{ansi-black-intense}{\# 561 parameters}
) \textcolor{ansi-black-intense}{                  \# Total: 6 arrays, }189\_457 parameters, 740.441
KiB.
\end{Verbatim}
\end{tcolorbox}
        
    \begin{tcolorbox}[breakable, size=fbox, boxrule=1pt, pad at break*=1mm,colback=cellbackground, colframe=cellborder]
\prompt{In}{incolor}{177}{\boxspacing}
\begin{Verbatim}[commandchars=\\\{\}]
\PY{c}{\PYZsh{} Define the loss function}
\PY{n}{loss}\PY{p}{(}\PY{n}{x}\PY{p}{,} \PY{n}{y}\PY{p}{)} \PY{o}{=} \PY{n}{Flux}\PY{o}{.}\PY{n}{crossentropy}\PY{p}{(}\PY{n}{softmax}\PY{p}{(}\PY{n}{model}\PY{p}{(}\PY{n}{x}\PY{p}{)}\PY{p}{)}\PY{p}{,} \PY{n}{Flux}\PY{o}{.}\PY{n}{onehotbatch}\PY{p}{(}\PY{n}{y}\PY{p}{,} \PY{l+m+mi}{1}\PY{o}{:}\PY{n}{length}\PY{p}{(}\PY{n}{label\PYZus{}mapping}\PY{p}{)}\PY{p}{)}\PY{p}{)}
\PY{c}{\PYZsh{} Define the optimizer}
\PY{n}{optimizer} \PY{o}{=} \PY{n}{Flux}\PY{o}{.}\PY{n}{ADAM}\PY{p}{(}\PY{p}{)}
\end{Verbatim}
\end{tcolorbox}

            \begin{tcolorbox}[breakable, size=fbox, boxrule=.5pt, pad at break*=1mm, opacityfill=0]
\prompt{Out}{outcolor}{177}{\boxspacing}
\begin{Verbatim}[commandchars=\\\{\}]
Adam(0.001, (0.9, 0.999), 1.0e-8, IdDict\{Any, Any\}())
\end{Verbatim}
\end{tcolorbox}
        
    \begin{tcolorbox}[breakable, size=fbox, boxrule=1pt, pad at break*=1mm,colback=cellbackground, colframe=cellborder]
\prompt{In}{incolor}{178}{\boxspacing}
\begin{Verbatim}[commandchars=\\\{\}]
\PY{c}{\PYZsh{} Train the model}
\PY{k}{for} \PY{n}{epoch} \PY{k}{in} \PY{l+m+mi}{1}\PY{o}{:}\PY{l+m+mi}{10}
    \PY{n}{Flux}\PY{o}{.}\PY{n}{train!}\PY{p}{(}\PY{n}{loss}\PY{p}{,} \PY{n}{Flux}\PY{o}{.}\PY{n}{params}\PY{p}{(}\PY{n}{model}\PY{p}{)}\PY{p}{,} \PY{n}{train\PYZus{}data}\PY{p}{,} \PY{n}{optimizer}\PY{p}{)}
\PY{k}{end}
\end{Verbatim}
\end{tcolorbox}

    \begin{tcolorbox}[breakable, size=fbox, boxrule=1pt, pad at break*=1mm,colback=cellbackground, colframe=cellborder]
\prompt{In}{incolor}{179}{\boxspacing}
\begin{Verbatim}[commandchars=\\\{\}]
\PY{c}{\PYZsh{} Step 4: Model Evaluation}
\PY{c}{\PYZsh{} Make predictions on the test set}
\PY{n}{X\PYZus{}test} \PY{o}{=} \PY{p}{[}\PY{n}{x} \PY{k}{for} \PY{p}{(}\PY{n}{x}\PY{p}{,} \PY{n}{\PYZus{}}\PY{p}{)} \PY{k}{in} \PY{n}{test\PYZus{}data}\PY{p}{]}
\PY{n}{y\PYZus{}test} \PY{o}{=} \PY{p}{[}\PY{n}{y} \PY{k}{for} \PY{p}{(}\PY{n}{\PYZus{}}\PY{p}{,} \PY{n}{y}\PY{p}{)} \PY{k}{in} \PY{n}{test\PYZus{}data}\PY{p}{]}
\PY{n}{y\PYZus{}pred} \PY{o}{=} \PY{n}{Flux}\PY{o}{.}\PY{n}{argmax}\PY{p}{(}\PY{n}{model}\PY{o}{.}\PY{p}{(}\PY{n}{X\PYZus{}test}\PY{p}{)}\PY{p}{,} \PY{n}{dims}\PY{o}{=}\PY{l+m+mi}{2}\PY{p}{)}
\end{Verbatim}
\end{tcolorbox}

            \begin{tcolorbox}[breakable, size=fbox, boxrule=.5pt, pad at break*=1mm, opacityfill=0]
\prompt{Out}{outcolor}{179}{\boxspacing}
\begin{Verbatim}[commandchars=\\\{\}]
582-element Vector\{Int64\}:
   1
   2
   3
   4
   5
   6
   7
   8
   9
  10
  11
  12
  13
   ⋮
 571
 572
 573
 574
 575
 576
 577
 578
 579
 580
 581
 582
\end{Verbatim}
\end{tcolorbox}
        
    \begin{tcolorbox}[breakable, size=fbox, boxrule=1pt, pad at break*=1mm,colback=cellbackground, colframe=cellborder]
\prompt{In}{incolor}{180}{\boxspacing}
\begin{Verbatim}[commandchars=\\\{\}]
\PY{c}{\PYZsh{} Calculate accuracy}
\PY{n}{accuracy} \PY{o}{=} \PY{n}{sum}\PY{p}{(}\PY{n}{y\PYZus{}pred} \PY{o}{.==} \PY{n}{reshape}\PY{p}{(}\PY{n}{y\PYZus{}test}\PY{p}{,} \PY{o}{:}\PY{p}{)}\PY{o}{\PYZsq{}}\PY{p}{)} \PY{o}{/} \PY{n}{length}\PY{p}{(}\PY{n}{y\PYZus{}test}\PY{p}{)}

\PY{n}{println}\PY{p}{(}\PY{l+s}{\PYZdq{}}\PY{l+s}{Accuracy: }\PY{l+s+si}{\PYZdl{}accuracy}\PY{l+s}{\PYZdq{}}\PY{p}{)}
\end{Verbatim}
\end{tcolorbox}

    \begin{Verbatim}[commandchars=\\\{\}]
Accuracy: 1.0
    \end{Verbatim}

    \begin{tcolorbox}[breakable, size=fbox, boxrule=1pt, pad at break*=1mm,colback=cellbackground, colframe=cellborder]
\prompt{In}{incolor}{181}{\boxspacing}
\begin{Verbatim}[commandchars=\\\{\}]
\PY{n}{println}\PY{p}{(}\PY{l+s}{\PYZdq{}}\PY{l+s}{Dimension of X\PYZus{}encoded: }\PY{l+s}{\PYZdq{}}\PY{p}{,} \PY{n}{size}\PY{p}{(}\PY{n}{X\PYZus{}encoded}\PY{p}{)}\PY{p}{)}
\PY{n}{println}\PY{p}{(}\PY{l+s}{\PYZdq{}}\PY{l+s}{Dimension of X\PYZus{}standardized: }\PY{l+s}{\PYZdq{}}\PY{p}{,} \PY{n}{size}\PY{p}{(}\PY{n}{X\PYZus{}standardized}\PY{p}{)}\PY{p}{)}
\end{Verbatim}
\end{tcolorbox}

    \begin{Verbatim}[commandchars=\\\{\}]
Dimension of X\_encoded: (2909, 2918)
Dimension of X\_standardized: (2909, 2918)
    \end{Verbatim}

    \begin{tcolorbox}[breakable, size=fbox, boxrule=1pt, pad at break*=1mm,colback=cellbackground, colframe=cellborder]
\prompt{In}{incolor}{182}{\boxspacing}
\begin{Verbatim}[commandchars=\\\{\}]
\PY{c}{\PYZsh{} Step 5: Model Deployment}
\PY{c}{\PYZsh{} You can now use the trained model to make predictions on new exercise names}

\PY{n}{exercise\PYZus{}name} \PY{o}{=} \PY{l+s}{\PYZdq{}}\PY{l+s}{Wrist Roller}\PY{l+s}{\PYZdq{}} \PY{c}{\PYZsh{} Example exercise name for prediction}
\PY{n}{exercise\PYZus{}encoded} \PY{o}{=} \PY{n}{Flux}\PY{o}{.}\PY{n}{onehotbatch}\PY{p}{(}\PY{p}{[}\PY{n}{exercise\PYZus{}name}\PY{p}{]}\PY{p}{,} \PY{n}{sort}\PY{p}{(}\PY{n}{unique}\PY{p}{(}\PY{n}{X}\PY{p}{)}\PY{p}{)}\PY{p}{)} \PY{o}{|\PYZgt{}} \PY{k+kt}{Matrix}\PY{p}{\PYZob{}}\PY{k+kt}{Float32}\PY{p}{\PYZcb{}}
\PY{n}{predicted\PYZus{}body\PYZus{}part} \PY{o}{=} \PY{n}{Flux}\PY{o}{.}\PY{n}{argmax}\PY{p}{(}\PY{n}{model}\PY{p}{(}\PY{n}{exercise\PYZus{}encoded}\PY{p}{)}\PY{p}{,} \PY{n}{dims}\PY{o}{=}\PY{l+m+mi}{2}\PY{p}{)}\PY{p}{[}\PY{l+m+mi}{1}\PY{p}{]}

\PY{n}{inverse\PYZus{}label\PYZus{}mapping} \PY{o}{=} \PY{n}{invert}\PY{p}{(}\PY{n}{label\PYZus{}mapping}\PY{p}{)}
\PY{n}{predicted\PYZus{}body\PYZus{}part\PYZus{}label} \PY{o}{=} \PY{n}{inverse\PYZus{}label\PYZus{}mapping}\PY{p}{[}\PY{n}{predicted\PYZus{}body\PYZus{}part}\PY{p}{]}

\PY{n}{println}\PY{p}{(}\PY{l+s}{\PYZdq{}}\PY{l+s}{Predicted body part: }\PY{l+s+si}{\PYZdl{}predicted\PYZus{}body\PYZus{}part\PYZus{}label}\PY{l+s}{\PYZdq{}}\PY{p}{)}
\end{Verbatim}
\end{tcolorbox}

    \begin{Verbatim}[commandchars=\\\{\}, frame=single, framerule=2mm, rulecolor=\color{outerrorbackground}]
DimensionMismatch: layer Dense(2918 => 64, relu) expects size(input, 1) == 2918, but got 2909×1 Matrix\{Float32\}

Stacktrace:
 [1] \_size\_check(layer::Dense\{typeof(relu), Matrix\{Float32\}, Vector\{Float32\}\}, x::Matrix\{Float32\}, ::Pair\{Int64, Int64\})
   @ Flux C:\textbackslash{}Users\textbackslash{}LENOVO\textbackslash{}.julia\textbackslash{}packages\textbackslash{}Flux\textbackslash{}EHgZm\textbackslash{}src\textbackslash{}layers\textbackslash{}basic.jl:195
 [2] (::Dense\{typeof(relu), Matrix\{Float32\}, Vector\{Float32\}\})(x::Matrix\{Float32\})
   @ Flux C:\textbackslash{}Users\textbackslash{}LENOVO\textbackslash{}.julia\textbackslash{}packages\textbackslash{}Flux\textbackslash{}EHgZm\textbackslash{}src\textbackslash{}layers\textbackslash{}basic.jl:171
 [3] macro expansion
   @ C:\textbackslash{}Users\textbackslash{}LENOVO\textbackslash{}.julia\textbackslash{}packages\textbackslash{}Flux\textbackslash{}EHgZm\textbackslash{}src\textbackslash{}layers\textbackslash{}basic.jl:53 [inlined]
 [4] \_applychain
   @ C:\textbackslash{}Users\textbackslash{}LENOVO\textbackslash{}.julia\textbackslash{}packages\textbackslash{}Flux\textbackslash{}EHgZm\textbackslash{}src\textbackslash{}layers\textbackslash{}basic.jl:53 [inlined]
 [5] (::Chain\{Tuple\{Dense\{typeof(relu), Matrix\{Float32\}, Vector\{Float32\}\}, Dense\{typeof(relu), Matrix\{Float32\}, Vector\{Float32\}\}, Dense\{typeof(identity), Matrix\{Float32\}, Vector\{Float32\}\}\}\})(x::Matrix\{Float32\})
   @ Flux C:\textbackslash{}Users\textbackslash{}LENOVO\textbackslash{}.julia\textbackslash{}packages\textbackslash{}Flux\textbackslash{}EHgZm\textbackslash{}src\textbackslash{}layers\textbackslash{}basic.jl:51
 [6] top-level scope
   @ In[182]:4
    \end{Verbatim}


    % Add a bibliography block to the postdoc
    
    
    
\end{document}
